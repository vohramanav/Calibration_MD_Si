
%%%%%%%%%%%%%%%%%%%%%%%%%%%%%%%%%%%%%%%%%%%%%%%%%%%%%%%%%%%%
%%  This Beamer template was created by Cameron Bracken.
%%  Anyone can freely use or modify it for any purpose
%%  without attribution.
%%
%%  Last Modified: January 9, 2009
%%

\documentclass[xcolor={x11names,table},compress,svgnames,mathserif]{beamer}

%% General document %%%%%%%%%%%%%%%%%%%%%%%%%%%%%%%%%%
\usepackage{graphicx}
\usepackage{tikz}
\usetikzlibrary{decorations.fractals}
%\usepackage{lmodern}
\usepackage{animate}
\usepackage{movie15}
\usepackage{bm}
\usepackage{pifont}
\usepackage{empheq}
\usepackage[many]{tcolorbox}
\usepackage{smartdiagram}
\usepackage[customcolors]{hf-tikz}
\usepackage{dashrule} % for dotted vertical line
%\usepackage{flexisym}

\usepackage[absolute,overlay]{textpos}

\pgfdeclarehorizontalshading{section shading}{2cm}{
color(0cm)=(LightSlateGrey);
color(2cm)=(gray!7);
color(3cm)=(LightSlateGrey!15)
}

% Custom block environment
% Custom block environment
\newenvironment<>{varblock}[2][.9\textwidth]{%
  \setlength{\textwidth}{#1}
  \begin{actionenv}#3%
    \def\insertblocktitle{#2}%
    \par%
    \usebeamertemplate{block begin}}
  {\par%
    \usebeamertemplate{block end}%
  \end{actionenv}}

% boxed equaton
\usepackage[many]{tcolorbox}
%\tcbuselibrary{skins}

\tcbset{highlight math style={enhanced,
  colframe=red!60!black,colback=yellow!50!white,arc=4pt,boxrule=1pt,
  }}

\newtcbox{\mybox}[1][]{nobeforeafter,math upper,tcbox raise base,
  enhanced,frame hidden,boxrule=0pt,interior style={top color=green!10!white,
  bottom color=green!10!white,middle color=green!50!yellow},
  fuzzy halo=1pt with green,drop large lifted shadow,#1}

%%%%%%%%%%%%%%%%%%%%%%%%%%%%%%%%%%%%%%%%%%%%%%%%%%%%%%

\usetikzlibrary{shapes,arrows}
\usetikzlibrary{positioning,decorations.pathreplacing}
% Define block styles
\tikzstyle{decision} = [diamond, draw, fill=purple!20, 
    text width=5.0em, text badly centered, node distance=3cm, inner sep=0pt]
\tikzstyle{block} = [rectangle, draw=none, fill=blue!20, anchor=north, 
    text width=9.0em, text centered]
    \tikzstyle{blockr} = [rectangle, draw, fill=blue!20, 
    text width=9.0em, text centered, rounded corners]
\tikzstyle{line} = [draw, -latex']
\tikzstyle{cloud} = [draw=none, ellipse,fill=purple!20, node distance=3cm,
    minimum height=2em]
    

%% Beamer Layout %%%%%%%%%%%%%%%%%%%%%%%%%%%%%%%%%%
\useoutertheme[subsection=false,shadow]{miniframes}
\useinnertheme{default}
%\usefonttheme{serif}
\usepackage{palatino}
\usepackage{xcolor}
\usepackage{amsmath}
\newcommand{\angstrom}{\textup{\AA}}

\setbeamerfont{title like}{shape=\scshape}
\setbeamerfont{frametitle}{shape=\scshape}
\setbeamertemplate{itemize items}[triangle] % if you wnat a circle
\setbeamertemplate{itemize subitem}[triangle]
\setbeamertemplate{navigation symbols}{}
%\setbeamertemplate{footline}[frame number]

\setbeamercolor{footlinecolor}{fg=white,bg=DeepSkyBlue4}
\defbeamertemplate*{footline}{infolines theme}
{
  \leavevmode%
  \hbox{%
  \begin{beamercolorbox}[wd=.333333\paperwidth,ht=2.25ex,dp=1ex,center]{footlinecolor}
 % {author in head/foot}%
    \usebeamerfont{author in head/foot}\insertshortauthor
   % \usebeamerfont{author in head/foot}\insertshortauthor~~(\insertshortinstitute)
  \end{beamercolorbox}%
  \begin{beamercolorbox}[wd=.333333\paperwidth,ht=2.25ex,dp=1ex,center]{title in head/foot}%
    \usebeamerfont{title in head/foot}\insertshorttitle
  \end{beamercolorbox}%
  \begin{beamercolorbox}[wd=.333333\paperwidth,ht=2.25ex,dp=1ex,right]{footlinecolor}
  %{date in head/foot}%
   % \usebeamerfont{date in head/foot}\insertshortdate{}\hspace*{2em}
    \usebeamerfont{date in head/foot}{manav.vohra@vanderbilt.edu}\hspace*{2em}
    \insertframenumber{} / \inserttotalframenumber\hspace*{2ex} 
  \end{beamercolorbox}}%
  \vskip0pt%
}
%\setbeamercolor{section in head/foot}{fg=white, bg=DeepSkyBlue4}

\setbeamercolor*{lower separation line head}{bg=DeepSkyBlue4} 
\setbeamercolor*{normal text}{fg=black,bg=white} 
\setbeamercolor*{alerted text}{fg=red} 
\setbeamercolor*{example text}{fg=black} 
\setbeamercolor*{structure}{fg=black} 
 
\setbeamercolor*{palette tertiary}{fg=black,bg=black!10} 
\setbeamercolor*{palette quaternary}{fg=black,bg=black!10} 

\renewcommand{\(}{\begin{columns}}
\renewcommand*\footnoterule{}
\renewcommand{\)}{\end{columns}}
\newcommand{\<}[1]{\begin{column}{#1}}
\renewcommand{\>}{\end{column}}

\newcommand*\subitem{%
  \item[\color{DeepSkyBlue4}\scalebox{0.6}{\ding{228}}]}
  
  \newcommand*\subitemtwo{%
  \item[\color{LightSlateGrey!15}\scalebox{0.6}{\ding{228}}]}

\newcommand*\myitem{%
  \item[\color{DeepSkyBlue4}\scalebox{0.6}{\ding{110}}]}
 
  \newcommand*\myitemtwo{%
  \item[\color{LightSlateGrey!15}\scalebox{0.6}{\ding{110}}]}
  
\newcommand*\Myitem{%
  \item[\color{DeepSkyBlue4}\scalebox{0.9}{\ding{42}}]}
  
\newcommand{\be}{\begin{equation}}
\newcommand{\ee}{\end{equation}}
\newcommand{\bea}{\begin{eqnarray}}
\newcommand{\eea}{\end{eqnarray}}
\newcommand{\p}{\partial}
\def\ol{\overline}
\def\no{\noindent}
\def\Vb{{\cal V}}
\def\Qd{\dot{Q}}
%\DeclareMathSymbol{\ast}{\mathbin}{symbols}{"03}
 \newcommand{\argmax}{\operatornamewithlimits{arg\,max}}

%-----------Parameter screening tikz styles -----------------------
   \tikzstyle{A} = [rectangle, rounded corners, minimum width=1.5cm, minimum height=0.5cm,text centered, draw=black,
     fill=blue!30]
      \tikzstyle{B} = [rectangle, rounded corners, minimum width=1.5cm, minimum height=0.5cm,text centered, draw=black,
     fill=red!30]
      \tikzstyle{p} = [rectangle, rounded corners, minimum width=1.5cm, minimum height=0.5cm,text centered, draw=black,
     fill=yellow!30]
      \tikzstyle{q} = [rectangle, rounded corners, minimum width=1.5cm, minimum height=0.5cm,text centered, draw=black,
     fill=green!30]
      \tikzstyle{alpha} = [rectangle, rounded corners, minimum width=1.5cm, minimum height=0.5cm,text centered, draw=black,
     fill=magenta!30]
      \tikzstyle{lambda} = [rectangle, rounded corners, minimum width=1.5cm, minimum height=0.5cm,text centered, draw=black,
     fill=cyan!30]
      \tikzstyle{gamma} = [rectangle, rounded corners, minimum width=1.5cm, minimum height=0.5cm,text centered, draw=black,
     fill=orange!30]
 \tikzstyle{outer} = [rectangle, minimum width=1.5cm, minimum height=0.5cm,text centered, draw=black,dashed]    
 \tikzstyle{pool} = [rectangle, minimum width=2.0cm, minimum height=1.6cm,line width=1pt,
text centered, draw=black,dashed]    

%% Algorithm %%%%%
\usepackage[linesnumbered]{algorithm2e}
\newcommand{\nosemic}{\renewcommand{\@endalgocfline}{\relax}}% Drop semi-colon ;
\newcommand{\dosemic}{\renewcommand{\@endalgocfline}{\algocf@endline}}% Reinstate semi-colon ;
\newcommand{\pushline}{\Indp}% Indent
\newcommand{\popline}{\Indm\dosemic}% Undent
\let\oldnl\nl% Store \nl in \oldnl
\newcommand{\nonl}{\renewcommand{\nl}{\let\nl\oldnl}}% Remove line number for one
\SetKwRepeat{Do}{do}{while}
%%%%%%%%%%%%%%%%%
 
%---------------------QUOTATION--------------------------
\usepackage{etoolbox}
%\usepackage[svgnames]{xcolor}
\usepackage{framed}

% conditional for xetex or luatex
\newif\ifxetexorluatex
\ifxetex
  \xetexorluatextrue
\else
  \ifluatex
    \xetexorluatextrue
  \else
    \xetexorluatexfalse
  \fi
\fi
%
\ifxetexorluatex%
  \usepackage{fontspec}
  \usepackage{libertine} % or use \setmainfont to choose any font on your system
  \newfontfamily\quotefont[Ligatures=TeX]{Linux Libertine O} % selects Libertine as the quote font
\else
  \usepackage[utf8]{inputenc}
  \usepackage[T1]{fontenc}
  %\usepackage{libertine} % or any other font package
  \newcommand*\quotefont{\fontfamily{LinuxLibertineT-LF}} % selects Libertine as the quote font
\fi

\newcommand*\quotesize{30} % if quote size changes, need a way to make shifts relative
% Make commands for the quotes
\newcommand*{\openquote}
   {\tikz[remember picture,overlay,xshift=-3ex,yshift=-0.5ex]
   \node (OQ) {\quotefont\fontsize{\quotesize}{\quotesize}\selectfont``};\kern0pt}

\newcommand*{\closequote}[1]
  {\tikz[remember picture,overlay,xshift=-15ex,yshift={#1}]
   \node (CQ) {\quotefont\fontsize{\quotesize}{\quotesize}\selectfont''};}

% select a colour for the shading
\colorlet{shadecolor}{Azure}

\newcommand*\shadedauthorformat{\emph} % define format for the author argument

% Now a command to allow left, right and centre alignment of the author
\newcommand*\authoralign[1]{%
  \if#1l
    \def\authorfill{}\def\quotefill{\hfill}
  \else
    \if#1r
      \def\authorfill{\hfill}\def\quotefill{}
    \else
      \if#1c
        \gdef\authorfill{\hfill}\def\quotefill{\hfill}
      \else\typeout{Invalid option}
      \fi
    \fi
  \fi}
% wrap everything in its own environment which takes one argument (author) and one optional argument
% specifying the alignment [l, r or c]
%
\newenvironment{shadequote}[2][l]%
{\authoralign{#1}
\ifblank{#2}
   {\def\shadequoteauthor{}\def\yshift{-2ex}\def\quotefill{\hfill}}
   {\def\shadequoteauthor{\par\authorfill\shadedauthorformat{#2}}\def\yshift{2ex}}
\begin{snugshade}\begin{quote}\openquote}
{\shadequoteauthor\quotefill\closequote{\yshift}\end{quote}\end{snugshade}}


%%%%%%%%%%%%%%%%%%%%%%%%%%%%%%%%%%%%%%%%%%%%%%%%%%

\title[UQ: Nano-scale Phonon Heat Transfer]{\textbf{Characterizing the Uncertainties in Non-Equilibrium MD for Thermal Transport}}
\thispagestyle{empty}
\pgfsetfillopacity{0.9}
\setbeamercolor{title}{bg=DeepSkyBlue4,fg=white}


%\subtitle{SUBTITLE}
\author[M. Vohra and S. Mahadevan]{Manav Vohra$^{\dag}$, Sankaran Mahadevan$^{\dag}$ \\ \vspace{2mm}
\scriptsize{Collaborators: Seungha Shin$^{\S}$, Ali Yousefzadi Nobakht$^{\S}$, Alen 
Alexanderian$^{\ddag}$}}
\vspace{-1mm}
\institute{$^{\dag}$Vanderbilt University\\ \vspace{1mm}
$^{\S}$University of Tennessee, Knoxville\\ \vspace{1mm}
$^{\ddag}$North Carolina State University, Raleigh}

\date{\today}


\begin{document}


%___________________________NEW SLIDE______________________________________
{
\setbeamertemplate{headline}{}

\begin{frame}[noframenumbering]

\titlepage
\vspace{-21mm}
\centering
%\scriptsize Institute for Computational Engineering and Sciences\\ \vspace{1mm}
%The University of Texas at Austin \\ \vspace{6mm}

%\footnotesize Seminar at SCI, University of Utah \\ \vspace{2mm}


\end{frame}
}

%___________________________NEW SLIDE______________________________________

\section{\scshape Background}
\subsection{md}
\begin{frame}{Background}

\begin{itemize}

\myitem Classical MD is used to investigate heat transfer dominated by phonon-phonon interactions
in material systems.  
\begin{itemize}
	\item Commonly applied to study non-metallic systems like C, Si, and Ge. 
\end{itemize}
		\vspace{1mm}
\myitem Typically conducted under equilibrium conditions characterized by thermodynamic ensembles like 
	NVT, NVE, NPT, and $\mu$VT.
		\vspace{1mm}

\myitem Non-Equilibrium MD involves setting up thermostats in different regions to establish temperature
gradients.
		\begin{itemize} 
			\item Thermostatting introduces errors. 
		\end{itemize}
		\vspace{1mm}
\end{itemize}

	{\scshape Why MD$?$} 
	\begin{itemize}
			\Myitem Enables simulation of much larger systems compared to DFT 
			in a reasonable amount of time. \vspace{1mm}
			\Myitem Trends from MD are useful despite possible errors in estimates.  
	\end{itemize}


\end{frame}

%___________________________NEW SLIDE______________________________________

\section{\scshape Plan}
\subsection{plan}
\begin{frame}{Plan}


Part I: \textsc{Forward Problem}:

\begin{itemize}

\myitem Investigate error in predictions due to {\color{blue}size} of the material system and
{\color{blue}fluctuations} in thermal gradient.

\begin{itemize} 
\item Construct a {\color{blue}response surface} for the error. 
\end{itemize}

%\vspace{2mm}

\myitem Characterize the impact of uncertainty in inter-atomic potential on predictions. 

\begin{itemize}
\item Dimension reduction using derivative-based sensitivity measures. 
\item Construction of PCE in the reduced space.
\item Forward Propagation of uncertainty. 
\end{itemize}

\end{itemize}

Part II: \textsc{Inverse Problem}:

\begin{itemize}

\myitem Calibrate critical parameters associated with the potential function in
a Bayesian setting using experimental data.

%\begin{itemize} 
%\item Exploit the error response surface from Part I.  
%\end{itemize}

\end{itemize}

\end{frame}

%___________________________NEW SLIDE______________________________________

\section{\scshape Set-up}
\subsection{setup}
\begin{frame}{NEMD on a Silicon Bar}

\begin{columns}
	\begin{column}{0.5\textwidth}

\begin{center}
\begin{tikzpicture}
\node[fill=white,inner sep=0pt] 
{\rowcolors{1}{DeepSkyBlue!20}{DeepSkyBlue!5}
	\resizebox{\columnwidth}{!}{
  \begin{tabular}{|l|c|}\hline
	  Lattice Constant, $a$ ($\angstrom$) & 5.43 \\
	  $W$, $H$ ($\angstrom$) & 22$a$, 22$a$ \\
	  Temperature (K) & 300 \\
	  $\Delta t$ (ps) & 0.0005 \\
	  BC & Periodic \\
	  Structure & Diamond \\
	  Potential & Stillinger-Weber \\
  \hline
  \end{tabular}%
	}
};
\end{tikzpicture}
\end{center}

\end{column}

\begin{column}{0.5\textwidth}

\begin{center}
%        \animategraphics[autopause,poster=first,loop,height=0.60\textwidth]{5}
%        {./animation/snap}{0}{64}
\begin{figure}[htbp]
        \includegraphics[width=0.9\textwidth]{./Figures/snap0}
\\
\tiny {$L$ = 50$a$, $N$ $\approx$ 200000 atoms}
\end{figure}

\end{center}

\vspace{-5mm}


\begin{figure}[htbp]
	\includegraphics[width=0.9\textwidth]{./Figures/temp_plot}
\end{figure}

\end{column}
\end{columns}

\end{frame}

%___________________________NEW SLIDE______________________________________

\subsection{setup}
\begin{frame}{Direct Method}

%\vspace{-4mm}
\begin{center}

{\color{green}NVT} \hspace{5mm} $\rightarrow$ \hspace{5mm} {\color{cyan}NVE} \hspace{5mm}
$\rightarrow$ \hspace{5mm} {\color{magenta}NVE}
\\ \vspace{1mm}
\tiny \hspace{-5mm}[Eqilibrate system to 300 K] \hspace{1mm} [Equilibrate thermostats] \hspace{4mm}
 [Generate Data]
\\ \vspace{1mm}

\tiny{N: Number of Atoms~~~V: Volume~~~T: Temperature~~~E: Energy}
\end{center}

\normalsize
\textsc{Observable}: Average energy exchange b/w thermostats ($q$)
\\ \vspace{2mm}
\textsc{QoI}: Bulk thermal conductivity ($\kappa$)

\begin{empheq}[box=\tcbhighmath]{align}
  \kappa = \frac{q}{\left|\frac{dT}{dx}\right|} \nonumber
\end{empheq}


\vspace{3mm}
\textsc{Initial Runs}:

\begin{itemize}
\myitem Determine the time steps needed for equilibration. 
\vspace{1mm}
\myitem Select a {\color{blue}reasonable} width and height for the Si bar.
\begin{itemize}
\item Small fluctuations due to changes in height and width.    
\end{itemize}

\end{itemize}

\end{frame}


%___________________________NEW SLIDE______________________________________

\subsection{setup}
\begin{frame}{Selection of Width}

	\begin{columns}
	\begin{column}{0.5\textwidth}
		\begin{figure}[htbp]
			\includegraphics[width=0.95\textwidth]{./Figures/nvt_L50_wh_effect}
	%		\\
	%		\includegraphics[width=0.85\textwidth]{./Figures/nve_L50_wh_effect}
			
		\end{figure}

	\end{column}
	\begin{column}{0.5\textwidth}
		\begin{figure}[htbp]
			\includegraphics[width=0.95\textwidth]{./Figures/nve_L50_wh_effect}
	%		\\
	%		\includegraphics[width=0.85\textwidth]{./Figures/nve_L100_wh_effect}

		\end{figure}

	\end{column}
	
	\end{columns}

\vspace{2mm}
\begin{itemize}
\myitem Norm of the fluctuations ($NF$) is computed using:

\begin{center}
\vspace{-5mm}

	\be 
            NF = \frac{1}{N}\left[\sum_k \left(T_k - T_{\{nvt,nve\}}\right)^{2}\right]^{\frac{1}{2}} \nonumber
        \ee

\end{center}
\end{itemize}

\end{frame}

%___________________________NEW SLIDE______________________________________

\subsection{setup}
\begin{frame}{Selection of Width}

	\begin{columns}
	\begin{column}{0.5\textwidth}
		\begin{figure}[htbp]
			\includegraphics[width=1.0\textwidth]{./Figures/nvt_l_effect}
		\end{figure}

	\end{column}
	\begin{column}{0.5\textwidth}
		\begin{figure}[htbp]
			\includegraphics[width=1.0\textwidth]{./Figures/nve_l_effect}
		\end{figure}

	\end{column}
	
	\end{columns}

%\vspace{2mm}
\begin{itemize}

\myitem At $W$ = 22$L_c$, the effect of length on fluctuations seems negligible. 
\end{itemize}

\end{frame}

%___________________________NEW SLIDE______________________________________

\section{\scshape Error Response}
\subsection{err}
\begin{frame}{Need a Surrogate$?$}

\begin{columns}
\begin{column}{0.5\textwidth}

\begin{itemize}
\myitem \textsc{Objective}: Forward UQ, Sensitivity Analysis, calibration, Design
\vspace{2mm}
\myitem \textsc{Computational Effort}: Simulations are computationally intensive.
\vspace{2mm}
\myitem \textsc{Accuracy}: Can a surrogate represent the observable
with reasonable accuracy in the domain of interest$?$ 
\end{itemize}

\end{column}

\begin{column}{0.5\textwidth}

\begin{figure}[htbp]
	\includegraphics[width=1.0\textwidth]{./Figures/cond_size}
\end{figure}


\end{column}
\end{columns}

\end{frame}


%___________________________NEW SLIDE______________________________________

\subsection{err}
\begin{frame}{Model Realizations}

\begin{columns}

\begin{column}{0.5\textwidth}
\begin{center}
\begin{figure}[htbp]
  \includegraphics[width=1.2\textwidth]{./Figures/realz_quad300K}
\end{figure}
\end{center}
\end{column}
\hspace{-6mm}
\begin{column}{0.5\textwidth}
\begin{center}
\begin{figure}[htbp]
  \includegraphics[width=1.2\textwidth]{./Figures/realz_sobol}
\end{figure}
\end{center}
\end{column}
\end{columns}

%\vspace{1mm}
\begin{itemize}
\myitem Model realizations at Gauss-Legendre quadrature nodes are used to
construct the PC surrogate.
\end{itemize}

\end{frame}


%___________________________NEW SLIDE______________________________________

\subsection{err}
%
\begin{frame}{PC Expansion}
%
\begin{columns}
\begin{column}{0.5\textwidth}

\begin{center}
\begin{empheq}[box=\tcbhighmath]{align}
  \epsilon (= |\kappa_m - \kappa_{\mbox{\tiny{MD}}}|) = \sum_j c_j\Psi_j(\xi_1,\xi_2) \nonumber
\end{empheq}

\small {$\kappa_m$: \tiny{Measured}\small, $\kappa_{\mbox{\tiny{MD}}}$: 
\tiny{MD Prediction}\small, $j$: \tiny{Multi-index}}
\end{center}

\end{column}

\hspace{-5mm}
\begin{column}{0.5\textwidth}

\begin{figure}[htbp]
	\includegraphics[width=1.2\textwidth]{./Figures/PCspectrum_300}
\end{figure}


\end{column}
\end{columns}

\vspace{1mm}
\begin{center}
\normalsize
$L$:~$\mathcal{U}[50L_c,100L_c]~(\angstrom)~~\rightarrow~~\xi_1: \mathcal{U}[-1,1]$ \\ \vspace{2mm}
$\frac{dT}{dx}$:~$\mathcal{U}[1.5/L_c,2.5/L_c]~(\frac{K}{\angstrom})~~\rightarrow~~\xi_2: \mathcal{U}[-1,1]$
\end{center}

\end{frame}


%___________________________NEW SLIDE______________________________________

\subsection{err}
%
\begin{frame}{Response Surface: $\epsilon(L,\frac{dT}{dx})$}
%
\begin{columns}
\begin{column}{0.5\textwidth}
%\vspace{-8mm}
\textsc{Accuracy}:
\vspace{-2mm}

\begin{center}
\begin{figure}[htbp]
%\hspace{-5mm}
  \includegraphics[width=0.65\textwidth]{./Figures/err2D_s}
\end{figure}

\vspace{-2mm}
%\tiny{
%\begin{empheq}[box=1.2\tcbhighmath]{align}
%  \alpha = \frac{\left[ \sum_j \left(\mathcal{G}\left(L(\xi_{1_j}),\frac{dT}{dx}(\xi_{2_j})\right) - \sum_k c_k\Psi_k(\xi_{1_j},\xi_{2_j})\right)^2\right]^{\frac{1}{2}}}{\left[\sum_j\left(\mathcal{G}\left(L(\xi_{1_j}),\frac{dT}{dx}(\xi_{2_j})\right)\right)^2\right]^{\frac{1}{2}}} \nonumber
%\end{empheq}
%}
\tiny{
\begin{empheq}[box=\tcbhighmath]{align}
  \epsilon = \frac{\left[ \sum_j \left(\mathcal{G}_M - \mathcal{G}_{PCE})\right)^2\right]^{\frac{1}{2}}}{\left[\left(\mathcal{G}_{M}\right)^2\right]^{\frac{1}{2}}} \approx 1.8\times10^{-3} \nonumber
\end{empheq}
}

\vspace{-3mm}
\tiny{
$\mathcal{G}_M$: Model Output\hspace{5mm}$\mathcal{G}_{PCE}$: PCE Estimate}
\end{center}

\end{column}

\begin{column}{0.5\textwidth}
\vspace{-10mm}
\begin{center}
\begin{figure}[htbp]

	\includegraphics[width=0.8\textwidth]{./Figures/err2D_300}
	\\ \tiny (a) 300~K \\
	\includegraphics[width=0.8\textwidth]{./Figures/err2D_500}
	\\ \tiny (b) 500~K

\end{figure}
\end{center}


\end{column}
\end{columns}

\end{frame}

%___________________________NEW SLIDE______________________________________

\subsection{err}
%
\begin{frame}{Bulk Thermal Conductivity}

\begin{columns}

\begin{column}{0.5\textwidth}
\begin{center}
\begin{figure}[htbp]
  \includegraphics[width=0.95\textwidth]{./Figures/kinv_300}
  \\ \vspace{2mm} \tiny {$T$ = 300 K}
\end{figure}
\end{center}
\end{column}
%\hspace{-6mm}
\begin{column}{0.5\textwidth}
\begin{center}
\begin{figure}[htbp]
  \includegraphics[width=0.9\textwidth]{./Figures/kinv_1000}
  \\ \vspace{2mm} \tiny {$T$ = 1000 K}
\end{figure}
\end{center}
\end{column}
\end{columns}


\end{frame}

%___________________________NEW SLIDE______________________________________

\section{\scshape Sensitivity}
\subsection{sense}
%
\begin{frame}{Uncertainty in Inter-Atomic Potential}

\textsc{Objectives}

\begin{itemize}
\myitem Relative importance of potential field parameters using sensitivity analysis.
\vspace{1mm}
\myitem Assess variability in thermal conductivity estimates due to
perturbations in the potential field (Forward Problem). 

\be
\Phi(A,B,p,q,a,\lambda,\gamma) \mapsto k \nonumber
\ee

\vspace{1mm}
\myitem Robust calibration of the potential field parameters in a Bayesian setting.

\end{itemize}

\textsc{Challenges}

\begin{itemize}
\myitem Global sensitivity analysis and the forward problem are computationally intractable.
\vspace{1mm}
\myitem Explore the applicability of a derivative-based sensitivity measure to reduce the
dimensionality of the problem. 
\end{itemize}

\end{frame}


%___________________________NEW SLIDE______________________________________

\subsection{sense}
%
\begin{frame}{Derivative-based Global Sensitivity Measures}

\textsc{Motivation}

\begin{itemize}
\myitem Sensitivity analysis based on Sobol indices is commonly used to determine
relative importance of the parameters. 
\vspace{2mm}
\myitem Sobol sensitivity indices are compute intensive:
\begin{equation}
 \mathcal{T}(\theta_i) = \frac{\mathbb{E}_{\bm{\theta}\sim i}[\mathbb{V}_{\theta_i}(\mathcal{G}|\bm{\theta}_{\sim i})]}{\mathbb{V}(\mathcal{G})} \nonumber
\end{equation}
\vspace{2mm}
\myitem Bounds on Sobol indices can be computed easily using DGSM and are shown to converge at a 
much faster rate. 

\end{itemize}

\end{frame}

%___________________________NEW SLIDE______________________________________

\subsection{sense}
%
\begin{frame}{Derivative-based Global Sensitivity Measures}

\textsc{Background}

\begin{itemize}

\myitem DGSM for Randomly distributed parameters 
\footnotesize{[Sobol and Kucherenko, 2009]}:\vspace{2mm}

\normalsize
\be
\mu_i = \mathbb{E}\left[\left(\frac{\partial G(\bm{x})}{\partial x_i}\right)^{2}\right]
\nonumber
\ee

where,

\be
\frac{\partial G(\bm{x}^{\ast})}{\partial x_i} = \lim_{\delta\to 0}
\frac{[G(x_1^{*},\ldots,x_{i-1}^{*},x_i^{*}+\delta,x_{i+1}^{*},\ldots,x_d^{*}) - G(\bm{x}^{*})]}{\delta} \nonumber
\ee

\myitem Total number of model realizations required to compute $\mu_i$ using $N$ samples
is $N(d+1)$.


\end{itemize}

\end{frame}


%___________________________NEW SLIDE______________________________________

\subsection{sense}
%
\begin{frame}{Derivative-based Global Sensitivity Measures}

\textsc{Background}

\begin{itemize}

\myitem Upper bound on Sobol Total Effect index 
($ST_i$)~\footnotesize{[Sobol and Kucherenko, 2009]}:

\normalsize
\be
ST_i \leq \frac{\mathcal{C}_i\mu_i}{V}~(\propto \hat{\mathcal{C}_i\mu_i}) \nonumber
\ee

\be
\hat{\mathcal{C}_i\mu_i} = \frac{\mathcal{C}_i\mu_i}{\sum_i \mathcal{C}_i\mu_i} \nonumber
\ee
\begin{center}
\tiny {$\mathcal{C}$: Poincar\'e Constant\hspace{3mm}  $V$: Variance}
\end{center}

\normalsize
\vspace{2mm}
\myitem The Poincar\'e Constant is specific to a given probability distribution:

\vspace{1mm}
\begin{center}
{\renewcommand{\arraystretch}{1.5}
\begin{tabular}{|p{1.5cm}|p{1.5cm}|}
 \hline
\footnotesize $\mathcal{U}[a,b]$ & \footnotesize $(b-a)^{2}/\pi^2$ \\ \hline
\footnotesize $\mathcal{N}(\mu,\sigma^2)$ & \footnotesize $\sigma^2$ \\ \hline
\end{tabular}
}
\end{center}

\end{itemize}

\end{frame}

%%___________________________NEW SLIDE______________________________________
%
\subsection{sense}
\begin{frame}{Algorithm: Parameter Screening}
%\textsc{Algorithm}
%\vspace{10mm}
\begin{algorithm}[H]
\SetAlgoLined
%\nonl \scriptsize{\textsc{Part I: Parameter Screening}}
\scriptsize
%\nonl \textbf{\texttt{Part I: Parameter Screening}}\;
\texttt{Generate $n_1$ points in $\mathbb{R}^{d}$}\;
\texttt{Compute $UB_i$ for parameters $\theta_i$ 
using $n_1$ points}\;
\nonl\scriptsize\texttt{\color{blue}$\%~NF$ = $n_1(d+1)$,~$NF$:~Number of
model realizations}\;
%\normalsize
\texttt{Rank Parameters $(\theta_i)$ based on $UB_i$ estimates 
($\mathcal{R}^{\tiny {old}}$)}\;
\texttt{set $k$ = 1}
\scriptsize\texttt{\color{blue}$\%$~$k$:~Iteration counter}\;
%\normalsize
\Repeat{\texttt{$\mathcal{R}^{\tiny{new}}$ $\neq$ $\mathcal{R}^{\tiny{old}}$}}{
\texttt{Generate $\beta n_1$ new points in $\mathbb{R}^{d}$ ($\beta n_1\in\mathbb{Z}$)}\;
\texttt{Compute $UB_i^{\tiny{new}}$ using (1+$\beta k)n_1$ points}\;
\nonl\scriptsize\texttt{\color{blue}$\%~NF$ = $(1+\beta k)n_1(d+1)$}\;
%\normalsize
\texttt{Rank Parameters based on $UB_i^{\tiny{new}}$ estimates
($\mathcal{R}^{\tiny{new}}$)}\;
\If {(\texttt{$\mathcal{R}^{\tiny{new}}$ = $\mathcal{R}^{\tiny{old}}$})}{
\texttt{Compute: $r_i$ = $\frac{UB_i^{\tiny{new}}}{max(UB_i^{\tiny{new}})}$}\;
\texttt{Construct a set $s$ = $\{\theta_i~\ni~r_i<\alpha\}$}\;
\texttt{Exit the loop}\;
}
\texttt{set $k$ = $k$ + 1}\;
}
\texttt{Construct a validation set: ($\bm{\theta}_j$,$\mathcal{M}(\bm{\theta}_j)$),
$j$=1,2,$\ldots$,$NF$}\;
\end{algorithm}
\end{frame}

%%___________________________NEW SLIDE______________________________________
%
\subsection{sense}
\begin{frame}{MD Simulation}

\begin{center}

{\color{blue}NPT} \hspace{5mm} $\rightarrow$ \hspace{5mm} {\color{green}NVT} \hspace{5mm} $\rightarrow$ \hspace{5mm} {\color{cyan}NVE} \hspace{5mm} $\rightarrow$ \hspace{5mm} {\color{magenta}NVE}
\\ \vspace{1mm}
\tiny [Relax System Length] \hspace{2mm}[Equilibrate system to 300 K] \hspace{1mm} [Equilibrate thermostats] \hspace{4mm}
 [Generate Data]
\\ \vspace{1mm}

\tiny{N: Number of Atoms~~~P: Pressure~~~V: Volume~~~T: Temperature~~~E: Energy}
\end{center}


\begin{figure}[htbp]
   \includegraphics[width=0.65\textwidth]{./Figures/lx_npt}
\end{figure}

\end{frame}

%%___________________________NEW SLIDE______________________________________
%
\subsection{sense}
\begin{frame}{Structural Instability}

\begin{center}
        \animategraphics[autopause,poster=first,loop,height=0.45\textwidth]{5}
        {./anim_unstable/snap}{0}{49}
\end{center}

\end{frame}

%
%%___________________________NEW SLIDE______________________________________
%
\subsection{sense}
\begin{frame}{Parameter Screening}

\begin{center}
\begin{tikzpicture}
\hspace{-5mm}
\tiny
%\node (case_5) [outer]{$\hat{\mathcal{C}_i\mu_i}$: 5 Samples};
%\node (gamma5) [gamma,below of=case_5,yshift=0.2cm] {$\gamma$: 0.3018};
%\node (A5) [A, below of=gamma5,yshift=0.2cm] {$A$: 0.2089};
%\node (alpha5) [alpha, below of=A5,yshift=0.2cm] {$\alpha$: 0.1765};
%\node (p5) [p, below of=alpha5,yshift=0.2cm] {$p$: 0.1175};
%\node (q5) [q, below of=p5,yshift=0.2cm] {$q$: 0.1137};
%\node (lambda5) [lambda, below of=q5,yshift=0.2cm] {$\lambda$: 0.0443};
%\node (B5) [B, below of=lambda5,yshift=0.2cm] {$B$: 0.0373};
%%\node (pool5) [pool, below of=q5,yshift=-0.6cm]{};
%\hspace{-10mm}

%\node (case_10) [outer,right of=case_5,xshift=2.5cm]{$\hat{\mathcal{C}_i\mu_i}$:
%10 Samples};
\node (case_10) [outer]{$\hat{\mathcal{C}_i\mu_i}$: 10 Samples};
\node (gamma10) [gamma, below of=case_10,yshift=0.2cm] {$\gamma$: 0.2531};
\node (alpha10) [alpha, below of=gamma10,yshift=0.2cm] {$\alpha$: 0.2170};
\node (A10) [A, below of=alpha10,yshift=0.2cm] {$A$: 0.2014};
\node (B10) [B, below of=A10,yshift=0.2cm] {$B$: 0.0964};
\node (p10) [p, below of=B10,yshift=0.2cm] {$p$: 0.0916};
\node (q10) [q, below of=p10,yshift=0.2cm] {$q$: 0.0741};
\node (lambda10) [lambda, below of=q10,yshift=0.2cm] {$\lambda$: 0.0664};
%\node (pool) [pool, below of=B10,yshift=-0.6cm]{};

\hspace{-10mm}

\node (case_15) [outer,right of=case_10,xshift=2.5cm]{$\hat{\mathcal{C}_i\mu_i}$:
15 Samples};
\node (gamma15) [gamma, right of=gamma10,xshift=2.5cm] {$\gamma$: 0.2001};
\node (alpha15) [alpha, right of=alpha10,xshift=2.5cm] {$\alpha$: 0.1723};
\node (A15) [A, right of=A10,xshift=2.5cm] {$A$: 0.1508};
\node (B15) [B, right of=B10,xshift=2.5cm] {$B$: 0.1381};
\node (p15) [p, right of=p10,xshift=2.5cm] {$p$: 0.1253};
\node (q15) [q, right of=q10,xshift=2.5cm] {$q$: 0.1232};
\node (lambda15) [lambda, right of=lambda10,xshift=2.5cm] {$\lambda$: 0.0900};
%\node (pool) [pool, below of=gamma15,yshift=-1.4cm]{};

\hspace{-10mm}

\node (case_20) [outer,right of=case_15,xshift=2.5cm]{$\hat{\mathcal{C}_i\mu_i}$:
20 Samples};
\node (gamma20) [gamma, right of=gamma15,xshift=2.5cm] {$\gamma$: 0.1933};
\node (alpha20) [alpha, right of=alpha15,xshift=2.5cm] {$\alpha$: 0.1593};
\node (q20) [q, right of=A15,xshift=2.5cm] {$q$: 0.1468};
\node (lambda20) [lambda, right of=B15,xshift=2.5cm] {$\lambda$: 0.1325};
\node (A20) [A, right of=p15,xshift=2.5cm] {$A$: 0.1261};
\node (B20) [B, right of=q15,xshift=2.5cm] {$B$: 0.1242};
\node (p20) [p, right of=lambda15,xshift=2.5cm] {$p$: 0.1177};
\node (pool) [pool, below of=A20,yshift=-0.24cm]{};

\hspace{-10mm}

\node (case_25) [outer,right of=case_20,xshift=2.5cm]{$\hat{\mathcal{C}_i\mu_i}$:
25 Samples};
\node (gamma25) [gamma, right of=gamma20,xshift=2.5cm] {$\gamma$: 0.2075};
\node (alpha25) [alpha, right of=alpha20,xshift=2.5cm] {$\alpha$: 0.1495};
\node (q25) [q, right of=q20,xshift=2.5cm] {$q$: 0.1493};
\node (lambda25) [lambda, right of=lambda20,xshift=2.5cm] {$\lambda$: 0.1392};
\node (A25) [A, right of=A20,xshift=2.5cm] {$A$: 0.1217};
\node (B25) [B, right of=B20,xshift=2.5cm] {$B$: 0.1175};
\node (p25) [p, right of=p20,xshift=2.5cm] {$p$: 0.1153};
\node (pool) [pool, below of=A25,yshift=-0.24cm]{};

\hspace{-10mm}

\node (case_30) [outer,right of=case_25,xshift=2.5cm]{$\hat{\mathcal{C}_i\mu_i}$:
30 Samples};
\node (gamma30) [gamma, right of=gamma25,xshift=2.5cm] {$\gamma$: 0.2024};
\node (alpha30) [alpha, right of=alpha25,xshift=2.5cm] {$\alpha$: 0.1487};
\node (lambda30) [lambda, right of=q25,xshift=2.5cm] {$\lambda$: 0.1424};
\node (q30) [q, right of=lambda25,xshift=2.5cm] {$q$: 0.1404};
\node (A30) [A, right of=A25,xshift=2.5cm] {$A$: 0.1395};
\node (B30) [B, right of=B25,xshift=2.5cm] {$B$: 0.1142};
\node (p30) [p, right of=p25,xshift=2.5cm] {$p$: 0.1125};
\node (pool) [pool, below of=A30,yshift=-0.24cm]{};

\end{tikzpicture}
\end{center}

\end{frame}

%%___________________________NEW SLIDE______________________________________
%
\subsection{sense}
\begin{frame}{Parameter Screening}

\begin{center}
\begin{figure}[htbp]
        \includegraphics[width=0.65\textwidth]{./Figures/ub}
\end{figure}

\end{center}
\end{frame}


%%___________________________NEW SLIDE______________________________________
%
\section{\scshape PCE}
\subsection{pce}
\begin{frame}{PCE}

\begin{columns}
\begin{column}{0.5\textwidth}

%\begin{center}
\be
\kappa = \sum_{\bm{s}\in\mathcal{A}} c_{\bm{s}}\Psi_{\bm{s}}(\bm{\xi}) \nonumber
\ee

%\noindent where,
\scriptsize
\begin{center}
$\mathcal{A}$: Sparse basis 

\vspace{-3mm}
\be
 \bm{\xi}:~\{\xi_1(A),\xi_2(q),\xi_3(\alpha),
\xi_4(\lambda),\xi_5(\gamma)\} \nonumber
\ee

\vspace{-3mm}
\be
\Psi_{\bm{s}}(\bm{\xi}) = \psi_{s_1}(\xi_1)\psi_{s_2}(\xi_2)
\psi_{s_3}(\xi_3)\psi_{s_4}(\xi_4)\psi_{s_5}(\xi_5) \nonumber
\ee

\vspace{-3mm}
\be
\xi_i \sim \mathcal{U}[-1,1] \nonumber
\ee

\end{center}

\end{column}

\begin{column}{0.5\textwidth}

\begin{center}
\begin{figure}[htbp]
        \includegraphics[width=1.0\textwidth]{./Figures/PCE5D_eloo}
\end{figure}
\end{center}

\end{column}
\end{columns}

\end{frame}

%%___________________________NEW SLIDE______________________________________
%
\subsection{pce}
\begin{frame}{Verification}

\begin{columns}
\begin{column}{0.5\textwidth}

\textsc{L-2}

\tiny
\begin{align*}
\mbox{\normalsize{$\epsilon_{val}$}} &= \frac{\left[\sum_{i=1}^{N}\left(\mathcal{M}(\bm{x}^{(i)}) -
\mathcal{M}^{PCE}(\bm{x}^{(i)})\right)^{2}\right]^{\frac{1}{2}}}{\left[\sum_{i=1}^{N}
\left(\mathcal{M}(\bm{x}^{(i)})\right)^2\right]^{\frac{1}{2}}} \\ \vspace{2mm}
&= 6.88\times10^{-2}
\nonumber
\end{align*}

\end{column}

\begin{column}{0.5\textwidth}
\begin{center}
\begin{figure}[htbp]
        \includegraphics[width=1.0\textwidth]{./Figures/PCE5D_kde}
\end{figure}
\end{center}
\end{column}
\end{columns}
\end{frame}

%%___________________________NEW SLIDE______________________________________
%
\subsection{pce}
\begin{frame}{Sensitivity Analysis}

\begin{columns}
\begin{column}{0.5\textwidth}

\tiny
\begin{center}
\begin{tikzpicture}
\node (case_30) [outer]{$\hat{\mathcal{C}_i\mu_i}$:
30 Samples};
\node (gamma30) [gamma, below of=case_30,yshift=0.2cm] {$\gamma$: 0.2024};
\node (alpha30) [alpha, below of=gamma30,yshift=0.2cm] {$\alpha$: 0.1487};
\node (lambda30) [lambda, below of=alpha30,yshift=0.2cm] {$\lambda$: 0.1424};
\node (q30) [q, below of=lambda30,yshift=0.2cm] {$q$: 0.1404};
\node (A30) [A, below of=q30,yshift=0.2cm] {$A$: 0.1395};
\node (B30) [B, below of=A30,yshift=0.2cm] {$B$: 0.1142};
\node (p30) [p, below of=B30,yshift=0.2cm] {$p$: 0.1125};
\node (pool) [pool, below of=A30,yshift=-0.24cm]{};

\end{tikzpicture}
\end{center}


\end{column}

\begin{column}{0.5\textwidth}
\begin{center}

\textsc{Sobol Sensitivity Indices}
\vspace{5mm}

\begin{figure}[htbp]
        \includegraphics[width=1.0\textwidth]{./Figures/PCE5D_gsa}
\end{figure}
\end{center}
\end{column}
\end{columns}

\end{frame}

%%___________________________NEW SLIDE______________________________________
%
\section{\scshape Likelihood}
\subsection{like}
\begin{frame}{Likelihood}

\begin{columns}
\begin{column}{0.5\textwidth}

\scriptsize{Bayes' rule:}

\be
\mathcal{P}(\alpha,\gamma\vert \kappa) \propto \mathcal{P}(\kappa\vert
 \alpha,\gamma)\mathcal{P}(\alpha)\mathcal{P}(\gamma) \nonumber
\ee

\be
\mathcal{P}(\alpha) \sim \mathcal{U}[1.62,1.98] \nonumber
\ee

\be
\mathcal{P}(\gamma) \sim \mathcal{U}[1.08,1.32] \nonumber
\ee

\end{column}

\begin{column}{0.5\textwidth}
\begin{center}

\begin{figure}[htbp]
        \includegraphics[width=1.0\textwidth]{./Figures/gl}
\end{figure}
\end{center}
\end{column}
\end{columns}

\end{frame}

%%___________________________NEW SLIDE______________________________________
%
\section{\scshape Subspace}
\subsection{subspace}
\begin{frame}{Active Subspace}

\begin{columns}
\begin{column}{0.5\textwidth}

\scriptsize{Positive Semi-definite Matrix}:
\be
\mathcal{C} = \int \nabla f(\bm{x})\nabla f(\bm{x})^{T}\rho(\bm{x})d\bm{x} = \bm{W}\Lambda\bm{W}^T
\nonumber
\ee

%\vspace{1mm}
\scriptsize{Partition of Eigenpairs}:
\begin{gather}
\Lambda = 
\begin{bmatrix}
\Lambda_1 & \\ & \Lambda_2 \nonumber
\end{bmatrix},~~\bm{W} = \left[\bm{W}_1~\bm{W}_2\right]
\end{gather}
%\vspace{1mm}

\scriptsize{MC Integration}:
\be
\mathcal{C} \approx \hat{\mathcal{C}} = \frac{1}{N}\sum_{i=1}^{N} \nabla_{\bm{x}}f_i\nabla_{\bm{x}}f_i^T = 
\hat{\bm{W}}\hat{\Lambda}\hat{\bm{W}}^{T} \nonumber
\ee

\vspace{-1mm}
\be
f(\bm{x}) \approx g(\bm{W}_1^T\bm{x}) \nonumber
\ee

\end{column}

\begin{column}{0.5\textwidth}

\begin{center}
\begin{figure}[htbp]
        \includegraphics[width=1.0\textwidth]{./Figures/eig}
\end{figure}
\end{center}


\end{column}
\end{columns}
\end{frame}

%%___________________________NEW SLIDE______________________________________
%
\subsection{subspace}
\begin{frame}{1D Active Subspace}

\vspace{-2mm}
$f(\bm{x})~=~g(\eta_1^T\xi)~~(\bm{x}\rightarrow \xi\in[-1,1])$

\begin{columns}
\begin{column}{0.5\textwidth}

\begin{center}
\begin{figure}[htbp]
        \includegraphics[width=1.0\textwidth]{./Figures/ssp1D}
\end{figure}
\end{center}

\end{column}

\begin{column}{0.5\textwidth}

\begin{center}
\begin{figure}[htbp]
        \includegraphics[width=1.0\textwidth]{./Figures/pdf_comp_SSP1D}
\end{figure}
\end{center}


\end{column}
\end{columns}
\end{frame}

%%___________________________NEW SLIDE______________________________________
%
\subsection{subspace}
\begin{frame}{2D Active Subspace}

\vspace{-2mm}
$f(\bm{x})~=~g(\eta_1^T\xi,\eta_2^T\xi)$

\begin{columns}
\begin{column}{0.5\textwidth}

\begin{center}
\begin{figure}[htbp]
        \includegraphics[width=1.0\textwidth]{./Figures/ssp2D}
\end{figure}
\end{center}

\end{column}

\begin{column}{0.5\textwidth}

\begin{center}
\begin{figure}[htbp]
        \includegraphics[width=1.0\textwidth]{./Figures/ssp2D_2}
\end{figure}
\end{center}


\end{column}
\end{columns}
\end{frame}

%%___________________________NEW SLIDE______________________________________
%
\subsection{subspace}
\begin{frame}{3D Active Subspace}

\vspace{-2mm}
$f(\bm{x})~=~g(\eta_1^T\xi,\eta_2^T\xi,\eta_3^T\xi)$

\begin{columns}
\begin{column}{0.5\textwidth}

\begin{center}
\begin{figure}[htbp]
        \includegraphics[width=1.0\textwidth]{./Figures/pdf_comp_SSP3D}
\end{figure}
\end{center}

\end{column}

\begin{column}{0.5\textwidth}

\begin{center}
\begin{figure}[htbp]
        \includegraphics[width=1.0\textwidth]{./Figures/rn}
\end{figure}
\end{center}


\end{column}
\end{columns}
\end{frame}

\end{document}
