\section{Sensitivity Analysis of the Inter-atomic Potential}
\label{sec:sense}

As discussed earlier in Section~\ref{sec:intro}, bulk thermal conductivity estimates in NEMD
are dependent on the choice of the inter-atomic potential as well as values associated with
the individual potential parameters. In the case of silicon, the Stillinger-Weber inter-atomic potential
has been used for a wide variety of 
systems~(see~\cite{Laradji:1995,Zhang:2014,Jiang:2015,Watanabe:1999,Zhou:2013} and references therein).
However, according to Stillinger and Weber, the set of nominal
values as provided below in Table~2 is based on a constrained search in the 7D parameter space
to ensure structural stability and agreement with the available experimental data~\cite{Stillinger:1985}.

\begin{table}[htbp]
\begin{center}
\begin{tabular}{|c|c|c|c|c|c|c|}
\hline 
$A$ & $B$ & $p$ & $q$ & $\alpha$ & $\lambda$ & $\gamma$ \\
\hline \hline
7.049556277 & 0.6022245584 & 4.0 & 0.0 & 1.80 & 21.0 & 1.20 \\
\hline
\end{tabular}
\end{center}
\caption{Nominal values of the parameters of the Stillinger-Weber inter-atomic
potential~\cite{Stillinger:1985}.}
\end{table}

It is noteworthy that the underlying analysis which lead to these estimates of the nominal values did not
account for the presence of uncertainty due to
measurement error, noise inherent in MD predictions, inadequacies pertaining to the potential function,
and parametric uncertainties. It is therefore likely that the proposed nominal estimates could be 
improved depending upon the application. Hence, it is critical to understand the effects of variability in
SW potential parameters on bulk thermal conductivity predictions using NEMD. For this purpose, a possible
approach could involve a global sensitivity analysis of NEMD predictions on the SW potential parameters 
by estimating the so-called Sobol indices~\cite{Sobol:2001}. However, obtaining converged estimates of
Sobol indices typically requires tens of thousands of model evaluations. Since NEMD is compute-intensive,
estimating the Sobol indices directly would be impractical. Instead, we adopt a novel strategy that
aims to determine relative importance of the SW potential parameters based on derivative-based global 
sensitivity measures (DGSM)~\cite{Sobol:2010}. It is observed that for a given application, it might be possible to 
converge to the upper bound on
Sobol index with only a few iterations~($\mathcal{O}(10^{1})$) thereby leading to enormous computational savings. 
In that case, estimates of the upper bound could be used in lieu of the
Sobol indices to determine relative importance of the parameters. The upper bound on 
Sobol total effect index\footnote{Sobol total effect index is a measure of the contribution of a 
parameter to the variance of the observable while contribution from other parameters may or may not be 0.}
($\mathcal{T}_i$) can be expressed in terms of DGSM~($\mu_i$), the Poincar\' e 
constant~($\mathcal{C}_i$), and the total variance of the observed
quantity~($V$)~\cite{Lamboni:2013,Roustant:2014}:   

\be
\mathcal{T}_i \leq \frac{\mathcal{C}_i\mu_i}{V}~(\propto \hat{\mathcal{C}_i\mu_i}) 
\ee 

\noindent The derivative-based sensitivity measure, $\mu_i$ for a given parameter, $\theta_i$ is
defined as an expectation
of the derivative of the observable ($G(\bm{\theta})$) with respect to that parameter:

\be
\mu_i = \mathbb{E}\left[\left(\frac{\partial G(\bm{\theta})}{\partial \theta_i}\right)^{2}\right]
\label{eq:mu}
\ee

\noindent Latin hypercube sampling in the 7D parameter space is used to estimate $\mu_i$. Note that $G$ must 
exhibit a smooth variation with each parameter so that the derivative in Eq.~\ref{eq:mu} can be estimated
with reasonable accuracy either analytically or numerically. 
We define a normalized quantity, $\hat{\mathcal{C}_i\mu_i}$ to ensure that its summation over all parameters is 1:

\be
\hat{\mathcal{C}_i\mu_i} = \frac{\mathcal{C}_i\mu_i}{\sum_i \mathcal{C}_i\mu_i} 
\ee

\noindent The choice of $\mathcal{C}_i$ is specific to the marginal probability distribution of the uncertain model
parameter, $\theta_i$. 
%We consider all uncertain parameters to be uniformly
%distributed in the interval~$[a,b]$ in which case $\mathcal{C}_i$  is given as $(b-a)^{2}/\pi^2$~\cite{Roustant:2014}.
The underlying methodology for implementing DGSM to the present application involving thermal transport in bulk Si,
and our key findings are presented in the following section. 

\subsection{DGSM for SW potential parameters}
\label{sub:dgsm} 

We aim to compute the derivate-based sensitivity measure (DGSM) and hence the corresponding upper bound on the
Sobol total effect index~($\mathcal{T}_i$) for each parameter in the SW potential. For this purpose, we 
introduce small perturbations~($\mathcal{O}(10^{-5})n_i$; $n_i$ being the nominal value) to the nominal values
associated with each parameter and estimate the partial derivatives in Eq.~\ref{eq:mu} using finite difference. 
Hence, in order to compute $\mu_i$ using $N$ points in the d-dimensional parameter space, we require $N(d+1)$
model realizations. The SW potential parameters are considered to be uniformly distributed in a small interval
around the nominal value in which case $\mathcal{C}_i$ is given as $(u-l)^{2}/\pi^2$~\cite{Roustant:2014}; $u$
and $l$ being the upper and lower bounds of the interval respectively.   

Performing NEMD simulations using perturbed values of the SW potential parameters could however be challenging.
For certain combinations of the SW potential parameter values, the steady-state thermal
energy exchange between the thermostats is found to be non-physical at the end of the simulation. We believe that
this happened in situations where the structure had deviated too far from the equilibrium state and hence the
structural integrity of the bar is lost as illustrated in Figure~\ref{fig:dgsm1}(a). To avoid this
issue, we added an NPT ensemble prior to NVT in the NEMD simulation as shown in the following diagram:

\begin{center}

NPT \hspace{5mm} $\rightarrow$ \hspace{5mm} NVT \hspace{5mm} $\rightarrow$ \hspace{5mm} NVE \hspace{5mm}
$\rightarrow$ \hspace{5mm} NVE
\\ \vspace{1mm}
\tiny [Relax the system]~[Equilibrate system to 300 K] \hspace{1mm} [Equilibrate thermostats] \hspace{4mm}
 [Generate Data]
\\ \vspace{1mm}

\tiny{N: Number of Atoms~~~P: Pressure~~~V: Volume~~~T: Temperature~~~E: Energy}
\end{center}

\noindent The NPT stage of the simulation was allowed to continue for a sufficiently long duration to ensure
that the system is relaxed to a steady value of the bar length as shown in Figure~\ref{fig:dgsm1}(b).
The following algorithm provides the sequence of steps that were used to obtain approximate estimates of the
sensitivity measures for the SW potential parameters:
\bigskip

\texttt{Algorithm}

\begin{algorithm}[H]
\SetAlgoLined
%\nonl \scriptsize{\textsc{Part I: Parameter Screening}}
%\nonl \textbf{\texttt{Algorithm}}\;
\texttt{Generate $n_1$ points in $\mathbb{R}^{d}$}\;
\texttt{Perturb each point along the $d$ directions to obtain a set of $n_1(d+1)$ points}
\texttt{\color{blue} $\%$~$d$: Number of parameters in the SW potential i.e. 7}\;
\texttt{Compute $\mu_i$ using model evaluations at the $n_1(d+1)$ points in Eq.~\ref{eq:mu}}\;
\texttt{Determine initial ranks, $\mathcal{R}^{old}$ of the parameters based on $\hat{\mathcal{C}_i\mu_i}$ values}\;
\texttt{set $k$ = 1}
\texttt{\color{blue}$\%$~$k$:~Iteration counter}\;
\Repeat{\texttt{($\mathcal{R}^{\tiny{new}}$ $\neq$ $\mathcal{R}^{\tiny{old}}$ $\&$ 
$max\_pdev$~$>\tau$)~\color{blue}$\%$~$\tau$:~Tolerance}}{
\texttt{Generate $n_k$ new points in $\mathbb{R}^{d}$}\;
\texttt{Perturb each point along the $d$ directions to obtain a set of $n_k(d+1)$ points}\;
\texttt{Compute and store model evaluations at the $n_k(d+1)$ points}\;
\texttt{Compute $\mu_i$ using prior model evaluations at $(d+1)(n_1 + \sum_j^k n_j)$ points}\;
\texttt{Determine new ranks, $\mathcal{R}^{new}$ based on updated $\hat{\mathcal{C}_i\mu_i}$ values}\;
\texttt{Compute $max\_pdev$ = max($\frac{|\mu_{i,k} - \mu_{i,k-1}|}{ \mu_{i,k-1}}$)}
\texttt{\color{blue}$\%$~$max\_pdev$: Maximum percentage deviation in $\mu_i$ between successive iterations.}\;
\texttt{set $k$ = $k$ + 1}\;
}
\end{algorithm}

For the present application, we begin with $n_1$ = 10 samples in the 7D parameter space and add 5 points
at each iteration. Using a tolerance, $\tau$ = 0.05, the above algorithm took 4 iterations i.e. 25 points to 
provide approximate estimates for $\mu_i$. Since finite difference was used to estimate the derivates in
Eq.~\ref{eq:mu}, it required 25(7+1) i.e. 200 MD runs. It must be noted that although the computational effort
pertaining to the estimation of DGSM can be substantial it is nevertheless several orders of magnitude smaller
than directly estimating the Sobol indices as mentioned earlier. 
In Figure~\ref{fig:screen}, we plot $\hat{\mathcal{C}_i\mu_i}$ as obtained for the SW parameters at the 
end of 4 iterations. It appears that $\gamma$ is significantly more important that the rest, whereas NEMD
predictions are not so sensitive to $B$ and $p$. Large sensitivity towards $\gamma$ and $\alpha$ (cut-off radius)
 is in fact expected since  these two parameters impact the lattice constant directly. 


























