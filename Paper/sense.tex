\section{Sensitivity Analysis of the Inter-atomic Potential}
\label{sec:sense}

As discussed earlier in Section~\ref{sec:intro}, bulk thermal conductivity estimates in NEMD
are dependent on the choice of the inter-atomic potential as well as values associated with
the individual potential parameters. In the case of silicon, the Stillinger-Weber inter-atomic potential
has been used for a wide variety of 
systems~(see~\cite{Laradji:1995,Zhang:2014,Jiang:2015,Watanabe:1999,Zhou:2013} and references therein).
However, according to Stillinger and Weber, the set of nominal
values as provided below in Table~2 is based on a constrained search in the 7D parameter space
to ensure structural stability and agreement with the available experimental data~\cite{Stillinger:1985}.

\begin{table}[htbp]
\begin{center}
\begin{tabular}{|c||c|}
\hline
$A$ & 7.049556277 \\
$B$ & 0.6022245584 \\
$p$ & 4.0 \\
$q$ & 0.0 \\
$\alpha$ & 1.80 \\
$\lambda$ & 21.0 \\
$\gamma$ & 1.20 \\
\hline
\end{tabular}
\end{center}
\caption{Nominal values of the parameters of the Stillinger-Weber inter-atomic
potential~\cite{Stillinger:1985}.}
\end{table}

It is noteworthy that the underlying analysis which lead to these estimates did not account for the
presence of uncertainty due to
measurement error, noise inherent in MD predictions, inadequacies pertaining to the potential function,
and parametric uncertainties. It is therefore likely that the proposed nominal estimates could be 
improved depending upon the application. Hence, it is critical to understand the effects of variability in
SW potential parameters on buk thermal conductivity predictions using NEMD. For this purpose, a possible
approach could involve a global sensitivity analysis of NEMD predictions on the SW potential parameters 
by estimating the so-called Sobol indices~\cite{Sobol:2001}. However, obtaining converged estimates of
Sobol indices typically requires tens of thousands of model evaluations. Since NEMD is compute-intensive,
estimating the Sobol indices directly would be impractical. Instead, we adopt a novel strategy that
aims to determine relative importance of the SW potential parameters based on derivative-based global 
sensitivity measures (DGSM)~\cite{Sobol:2010}. In fact, it can be shown that the upper bound on a 
Sobol index can be expressed in terms of DGSM~($\mu_i$), the Poincare constant~($\mathcal{C}_i$), and the
total variance of the observed quantity~($V$)~\cite{Lamboni:2013,Roustant:2014}:    


