\section{Sensitivity Analysis of the Inter-atomic Potential}
\label{sec:sense}

As discussed earlier in Section~\ref{sec:intro}, bulk thermal conductivity estimates in NEMD
are dependent on the choice of the inter-atomic potential as well as values associated with
the individual potential parameters. In the case of silicon, the Stillinger-Weber inter-atomic potential
has been used for a wide variety of 
systems~(see~\cite{Laradji:1995,Zhang:2014,Jiang:2015,Watanabe:1999,Zhou:2013} and references therein).
However, according to Stillinger and Weber, the set of nominal
values as provided below in Table~2 is based on a constrained search in the 7D parameter space
to ensure structural stability and agreement with the available experimental data~\cite{Stillinger:1985}.

\begin{table}[htbp]
\begin{center}
\begin{tabular}{|c|c|c|c|c|c|c|}
\hline 
$A$ & $B$ & $p$ & $q$ & $\alpha$ & $\lambda$ & $\gamma$ \\
\hline \hline
7.049556277 & 0.6022245584 & 4.0 & 0.0 & 1.80 & 21.0 & 1.20 \\
\hline
\end{tabular}
\end{center}
\caption{Nominal values of the parameters of the Stillinger-Weber inter-atomic
potential~\cite{Stillinger:1985}.}
\end{table}

It is noteworthy that the underlying analysis which lead to these estimates of the nominal values did not
account for the presence of uncertainty due to
measurement error, noise inherent in MD predictions, inadequacies pertaining to the potential function,
and parametric uncertainties. It is therefore likely that the proposed nominal estimates could be 
improved depending upon the application. Hence, it is critical to understand the effects of variability in
SW potential parameters on bulk thermal conductivity predictions using NEMD. For this purpose, a possible
approach could involve a global sensitivity analysis of NEMD predictions on the SW potential parameters 
by estimating the so-called Sobol indices~\cite{Sobol:2001}. However, obtaining converged estimates of
Sobol indices typically requires tens of thousands of model evaluations. Since NEMD is compute-intensive,
estimating the Sobol indices directly would be impractical. Instead, we adopt a novel strategy that
aims to determine relative importance of the SW potential parameters based on derivative-based global 
sensitivity measures (DGSM)~\cite{Sobol:2010}. It is observed that for a given application, it might be possible to 
converge to the upper bound on
Sobol index with only a few iterations~($\mathcal{O}(10^{1})$) thereby leading to enormous computational savings. 
In that case, estimates of the upper bound could be used in lieu of the
Sobol indices to determine relative importance of the parameters. The upper bound on 
Sobol total effect index\footnote{Sobol total effect index is a measure of the contribution of a 
parameter to the variance of the observable while contribution from other parameters may or may not be 0.}
($\mathcal{T}_i$) can be expressed in terms of DGSM~($\mu_i$), the Poincar\' e 
constant~($\mathcal{C}_i$), and the total variance of the observed
quantity~($V$)~\cite{Lamboni:2013,Roustant:2014}:   

\be
\mathcal{T}_i \leq \frac{\mathcal{C}_i\mu_i}{V}~(\propto \hat{\mathcal{C}_i\mu_i}) 
\ee 

\noindent The derivative-based sensitivity measure, $\mu_i$ for a given parameter, $\theta_i$ is defined as an expectation
of the derivative of the observable ($G(\bm{\theta})$) with respect to that parameter:

\be
\mu_i = \mathbb{E}\left[\left(\frac{\partial G(\bm{\theta})}{\partial \theta_i}\right)^{2}\right]
\label{eq:mu}
\ee

\noindent Latin hypercube sampling in the 7D parameter space is used to estimate $\mu_i$. Note that $G$ must 
exhibit a smooth variation with each parameter so that the derivative in Eq.~\ref{eq:mu} can be estimated
with reasonable accuracy either analytically or numerically. 
 In this work, we examine the suitability of this
approach based on DGSM for the present application involving thermal transport in bulk Si. We define a 
normalized quantity, $\hat{\mathcal{C}_i\mu_i}$ to ensure that its summation over all parameters is 1:

\be
\hat{\mathcal{C}_i\mu_i} = \frac{\mathcal{C}_i\mu_i}{\sum_i \mathcal{C}_i\mu_i} 
\ee

\noindent The choice of $\mathcal{C}_i$ is specific to the marginal probability distribution of the uncertain model
parameter, $\theta_i$. In this work, we consider all uncertain parameters to be uniformly
distributed in the interval~$[a,b]$ in which case $\mathcal{C}_i$  is given as $(b-a)^{2}/\pi^2$~\cite{Roustant:2014}.

































