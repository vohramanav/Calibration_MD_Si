\section{Summary and Discussion}
\label{sec:disc}

In this paper, we have attempted to identify and address some of the challenges
pertaining to uncertainty analysis of bulk thermal conductivity predictions 
using non-equilibrium molecular dynamics (NEMD) simulations. Specifically, we focused
on investigating the impact of system size, and variability in the applied thermal
gradient on predictions. In order to quantify the discrepancy between NEMD
predictions and experiments, response surfaces were  
constructed at bulk temperatures, $T$ = 300~K, 500~K, and 1000~K.  
It was found that the discrepancy is predominantly impacted by size while the 
effect of variability in the applied thermal gradient is negligible in the considered
interval. The response surface approach presented here relies on a small number of
MD runs and enables an accurate estimation of discrepancy at a given temperature
and a point in the 2D parameter space described by system-size and the applied
thermal gradient. 

A possible enhancement of nominal SW parameter estimates for a given application
and the choice of material system is also highlighted in this work. To enable this,
we focus our efforts on understanding the sensitivity of predictions on the 
parameters. For this purpose, we estimate the so-called derivative based sensitivity
measures (DGSM) using random samples in the 7D parameter space. While individual
measures for the 7 parameters are not too distant from each other, the predictions
seem to be most sensitive towards $\gamma$. Sensitivity measures for $\alpha$,
$\lambda$, $q$, and $A$ are found to be comparable while those for $B$ and $p$
are relatively small. 

A polynomial chaos surrogate model for the bulk thermal conductivity (observable)
as a function of SW potential
parameters at a given temperature is constructed. The surrogate helps reduce the
computational effort required for forward propagation of the uncertainty from
parameters to the observable as well as for estimating the Sobol sensitivity indices.
Furthermore, the surrogate could be used to accelerate parameter calibration in a
Bayesian setting. However, since the surrogate relies on NEMD predictions, the
underlying computational effort is nevertheless significantly large. To circumvent
this challenge, we exploit our initial findings based on DGSM and construct
a reduced order surrogate in 5 dimensions by fixing the parameters, $B$ and $p$.
We verify its accuracy by estimating the relative
L-2 norm of the error between NEMD predictions in the full space and the reduced
order surrogate predictions, and comparing the probability distribution of the
bulk thermal conductivity in Figure~\ref{fig:verify}. Furthermore, our initial
sensitivity trends based on DGSM-analysis using 25 samples seems to agree favorably
with Sobol sensitivity analysis based on 10$^{6}$ samples. Hence, it can be said
that DGSM-based analysis with a few samples could offer huge computational gains
by reliably reducing the dimensionality of a surrogate for uncertainty analysis. 

Finally, we highlight key aspects of parameter calibration in a Bayesian setting.
The underlying motivation stems from the fact that the nominal estimates of the SW
parameters did not consider measurement error, simulation noise, model form error,
and parametric uncertainties. Calibration in a Bayesian framework allows us to
incorporate such errors and uncertainties in an efficient manner and provides
a joint posterior distribution of the uncertain parameters. To minimize computational
costs pertaining to the calibration process, we suggest the following sequence of
steps: First, perform DGSM analysis to 
identify parameter that are not so important. Second, construct a reduced order
surrogate based on DGSM-analysis. Third, verify the accuracy of the surrogate
and compute Sobol indices. Fourth, consider only the important parameters in
calibration and construct a second surrogate in the reduced subspace. Fifth,
quantify the surrogate error to be accounted for during calibration. Sixth,
evaluate the joint posterior using an efficient MCMC-based algorithm.

In conclusion, the authors would like to highlight that the strategies presented 
in this work are not restricted to a Si bar and the Stillinger-Weber potential, and
could be extended to a wide range of applications and inter-atomic potentials for
the purpose of uncertainty analysis. 

