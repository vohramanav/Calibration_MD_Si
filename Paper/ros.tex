\section{Reduced Order Surrogate}
\label{sec:ros}

In this section, we focus our attention on constructing a surrogate that captures the dependence of
uncertainty in NEMD predictions for the bulk thermal conductivity ($\kappa$) on input variability
in the SW potential parameters. The surrogate is a powerful tool that greatly minimizes
the computational
effort required for forward propagation of the uncertainty from input parameters to the observable,
Sobol sensitivity analysis, and Bayesian calibration of the uncertain model parameters. Once again,
we use polynomial chaos (used to construct response surfaces for discrepancy in
Section~\ref{sec:response}) to construct the surrogate of the following functional form:

\be
\kappa  = \sum\limits_{\bm{s}\in\mathcal{A}} c_{\bm{s}}(T)\Psi_{\bm{s}}(\bm{\xi})
\ee

As discussed earlier in Section~\ref{sec:response}, several strategies are available to
estimate the PC coefficients, $c_{\bm{s}}$. However, since the polynomial basis functions
($\Psi_{\bm{s}}(\bm{\xi}$)) are
relatively high dimensional, we use a computationally efficient approach proposed by
Blatman and Sudret to construct a PCE with sparse basis($\mathcal{A}$) using the LAR
algorithm~\cite{Blatman:2011}. Furthermore, since the NEMD simulations are compute-intensive,
estimating the PC coefficients in the 7D parameter space would still require large amount of
computational resources. Hence, we explore the possibility of reducing the dimensionality of
the surrogate. For this purpose, we 
exploit our observations in Figure~\ref{fig:ub} where a signifcant
jump in $\hat{\mathcal{C}_i\mu_i}$ estimate is seen from $A$ to $B$ and therby construct the 
PC surrogate in a 5D parameter space by fixing $B$ and $p$ at their nominal values. In the above
equation, $\bm{\xi}:~\{\xi_1(A),\xi_2(q),\xi_3(\alpha),\xi_4(\lambda),\xi_5(\gamma)\}$ is a set
of five canonical random variables, $\xi_i$ distributed uniformly in the interval [-1,1]. 
For the five uniformly distributed SW parameters, the prior intervals are considered to be
$\pm~10\%$ of their respective nominal estimates except for $q$ in which case it is [0,0.1]. In
Figure~\ref{fig:loo}, we plot the leave-one-out cross-validation error 
($\epsilon_{\tiny{\mbox{LOO}}}$)~\cite{Blatman:2010} as defined
below in Eq.~\ref{eq:loo} against the number of model realizations used to construct the 5D
PC surrogate. 

\be
\epsilon_{\tiny{\mbox{LOO}}} = \frac{\sum\limits_{i=1}^{N}\left(\mathcal{M}(\bm{x}^{(i)}) - 
\mathcal{M}^{PCE\setminus i}(\bm{x}^{(i)})\right)^{2}}{\sum\limits_{i=1}^{N}
\left(\mathcal{M}(\bm{x}^{(i)}) - \hat{\mu}_Y\right)^2}
\label{eq:loo}
\ee 

\noindent where $N$ denotes the number of realizations, $\mathcal{M}(\bm{x}^{(i)})$ is the
model realization and $\mathcal{M}^{PCE\setminus i}(\bm{x}^{(i)})$ is the corresponding PCE estimate
at $\bm{x}^{(i)}$. Note that the PCE is constructed using all points except $\bm{x}^{(i)}$.
The quantity, $\hat{\mu}_Y$ = $\frac{1}{N}\sum\limits_{i=1}^{N}\mathcal{M}(\bm{x}^{(i)})$
is the sample mean of the realizations. 
It is found that in order for the PCE to converge to an accuracy of~$\mathcal{O}(10^{-2})$
with respect to $\epsilon_{\tiny{\mbox{LOO}}}$, we require approximately 160 NEMD runs. In the
following section, we focus on verifying the accuracy of the 5D PCE against the set of available
NEMD predictions in the original 7D parameter space. 


