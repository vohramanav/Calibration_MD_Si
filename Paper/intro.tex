\section{Introduction}
\label{sec:intro}

\begin{comment}
1. Background on use of MD simulations for thermal transport, preferred for studying
thermal transport by phononic interactions (refer notes from book suggested by Amuthan)


2. One approach to NEMD is the Direct Method, commonly used for estimating the bulk
thermal conductivity. A brief discussion on the direct method and associated pros and cons
(notes from Dellan's paper and book suggested by Amuthan) 
Predictions impacted by the choice of potential, values of
individual parameters, size, and potentially due to duration and applied thermal gradients
(cite Amuthan book, Francesco's paper, McGaughey's paper). 
Errors are introduced by thermostatting (Amuthan book). Nominal value of SW potential parameters based on fitting against experiments and to ensure structural stability etc. (SW paper)

3. Motivate uncertainty analysis and briefly discuss and cite recent efforts (Francesco, Kirby,
Murthy). Highlight focus and key contributions of the present work and how it differs from
those efforts. 

4. Section-wise overview of the paper.  
\end{comment}

Classical Molecular Dynamics (MD) is commonly used to study thermal transport by means of
interactions between phonons in material systems comprising non-metallic elements such
as carbon, silicon, and germanium~\cite{Dumitrica:2010}. 
A major objective for many such studies is the
estimation of bulk thermal conductivity of the system. One of the most commonly used approach,
regarded as the direct method~\cite{Schelling:2002,Turney:2009,Zhou:2009,Landry:2009,
McGaughey:2006,Ni:2009,Shi:2009,Wang:2009,Papanikolaou:2008}, is a non-equilibrium
technique that involves the application
of a heat flux or a temperature gradient by means of thermostatting across the system. 
Corresponding steady-state temperature gradient in the former or the heat exchange between
the two thermostats in the latter, is used to estimate thermal conductivity (at a given
length or size) using 
Fourier's law. Computations are performed for a range of system lengths and the inverse
of thermal conductivity is plotted against the inverse of length. The y-intercept of a
straight line fit to the observed trend is considered as the bulk thermal conductivity 
estimate. 

Although widely used, the direct method is known to severely under-predict
the bulk thermal conductivity as compared to experimental 
measurements~\cite{Haynes:2014,Shanks:1963}. This is primarily
due to length scales used in simulation that are several orders of magnitude smaller
than those used in an experiment. As a result, the sample length is much smaller than the
bulk phonon mean free path leading to a so-called ballistic transport of the phonons.
The mean free path of such phonon modes is limited to the system size that reduces their
contribution to thermal transport. Moreover, the
introduction of thermostats typically reduces the correlation between vibrations of 
different atoms (regarded as Kapitza effect~\cite{Stevens:2007}) potentially reducing
the thermal conductivity
further~\cite{Evans:2008}. Estimate of thermal conductivity using the
direct method is therefore impacted by the choice of system size and potentially due to
variability in the thermal gradient experienced by the system owing to
Kapitza effect.  

Predictions of non-equilibrium molecular dynamics (NEMD) simulations are dependent on the
choice of the inter-atomic potential as well as values associated with individual parameters
of a given potential. For instance, in the case of crystalline Si, the Stillinger-Weber (SW)
is a widely used inter-atomic potential. However, as discussed by Stillinger and Weber
in~\cite{Stillinger:1985}, the proposed nominal values of individual parameters were based
on a limited search in the 7D parameter space while ensuring structural stability and
reasonable agreement with experiments. It is therefore likely that these nominal estimates
for individual parameter values in the SW potential may not yield accurate results for a
wide variety of Si-based systems and warrant further investigation to study the impact of
underlying uncertainties on MD predictions. 
In a recent effort, Rizzi et al. study the effect of
uncertainties associated with the force-field parameters on bulk water properties using MD
simulations~\cite{Rizzi:2012}. Marepalli et al. in~\cite{Marepalli:2014} consider a stochastic model
for thermal conductivity (due to noise in MD simulations) and study its impact on spatial temperature 
distribution during heat conduction. Jacobson et al. in~\cite{Jacobson:2014} implement an uncertainty
quantification framework to optimize a coarse-grained model for predicting the properties of monoatomic
 water. While these are significant contributions, it is only until recently 
that researchers have started accounting for the presence of uncertainties in MD predictions in a
systematic manner. There is a definite need for additional efforts aimed at efficiency and accuracy
to enable uncertainty analysis in MD simulations for a wide range of applications.

In the present work, we focus our efforts on uncertainty analysis in predictions of NEMD simulations
for phonon transport.  As discussed earlier, predictions from NEMD
exhibit large discrepancies with experimental observations depending upon system size and potentially
due to variability in the applied thermal gradient. Additionally, the thermal conductivity estimates are
tightly coupled with parameter values associated with the inter-atomic potential. Hence, we set out to
accomplish multiple objectives through this research effort: First, we construct response surfaces
in order to characterize the dependence of discrepancy in thermal conductivity estimates from NEMD
with the experimental data on system size and applied thermal gradient. Second, we perform
sensitivity analysis to study the impact of SW potential parameter values on variability in predictions.
Third, we exploit our findings from sensitivity analysis to discern parameters for calibration in a Bayesian
setting in order to enhance the predictability of NEMD simulations. 