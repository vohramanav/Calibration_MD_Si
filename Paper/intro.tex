\section{Introduction}
\label{sec:intro}

\begin{comment}
1. Background on use of MD simulations for thermal transport, preferred for studying
thermal transport by phononic interactions (refer notes from book suggested by Amuthan)


2. One approach to NEMD is the Direct Method, commonly used for estimating the bulk
thermal conductivity. A brief discussion on the direct method and associated pros and cons
(notes from Dellan's paper and book suggested by Amuthan) 
Predictions impacted by the choice of potential, values of
individual parameters, size, and potentially due to duration and applied thermal gradients
(cite Amuthan book, Francesco's paper, McGaughey's paper). 
Errors are introduced by thermostatting (Amuthan book). Nominal value of SW potential parameters based on fitting against experiments and to ensure structural stability etc. (SW paper)

3. Motivate uncertainty analysis and briefly discuss and cite recent efforts (Francesco, Kirby,
Murthy). Highlight focus and key contributions of the present work and how it differs from
those efforts. 

4. Section-wise overview of the paper.  
\end{comment}

Classical Molecular Dynamics (MD) is commonly used to study thermal transport by means of phonons in material 
systems comprising non-metallic elements such as carbon, silicon, germanium.  