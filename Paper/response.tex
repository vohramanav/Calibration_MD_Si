\section{Response Surface of the Discrepancy}
\label{sec:response}

As discussed earlier in Section~\ref{sec:intro}, bulk thermal conductivity estimates using NEMD simulations
are severely under-predicted primarily due to reduction in mean free path associated with phonon transport. 
Additionally, the introduction of thermostats causes a significant amount of variability in the applied thermal gradient,
especially in their vicinity owing to the Kapitza effect~\cite{Stevens:2007}. 
We illustrate this phenomenon by plotting temperature 
distribution along the length of the bar in Figure~\ref{fig:kapitza}. In this section, we focus on the impact of
system size, specifically the length of the Si bar as well as the variability in applied thermal gradient on discrepancy
in bulk thermal conductivity between NEMD predictions and experimental data. For this purpose, we consider
a range of values for the bar length and the thermal gradient. In order to determine the discrepancy trends, one
might consider evaluating the thermal conductivity using NEMD simulations for a large set of values of length and
thermal gradient. However, considering the computational expense associated with each set, this approach quickly
becomes computationally prohibitive. Instead, we construct a response surface using a 2D
Polynomial Chaos (PC) representation of the discrepancy which requires NEMD predictions for a small number of
combinations of length and thermal gradient values as discussed in the following section. 

\subsection{Polynomial Chaos}

Polynomial chaos (PC) exhibits a functional relationship between independent and uncertain 
parameters~$(\bm{\theta})$ 
in a model and a random observable, say $\mathcal{Y}$. Essentially, it is a truncated expansion with polynomial 
basis functions that converges in a least-squares sense. For an accurate PC representation, the observable should
vary smoothly with respect to the uncertain parameters~\cite{Vohra:2014}  and must
be L-2 integrable:

\be
\mathbb{E}[\mathcal{Y}^2] = \int_{\mathcal{D}_{\bm{\theta}}} \mathcal{Y}^2 \mathbb{P}(\bm{\theta}) 
d\bm{\theta} < \infty
\ee

\noindent where $\mathcal{D}_{\bm{\theta}}$ is the domain of the input parameter space and 
$\mathbb{P}(\bm{\theta})$ is the joint probability distribution of individual components of $\bm{\theta}$.
In the present setting, $\bm{\theta}$:~$\{L,\frac{dT}{dz}\}$ and the observable, $\mathcal{Y}$ is the
discrepancy~($\epsilon_{\mbox{\tiny{d}}}$ = 
$\lvert\kappa_{\mbox{\tiny{MD}}}$ - $\kappa_{\mbox{\tiny{E}}}\rvert$)
in bulk thermal conductivity predictions from 
NEMD~($\kappa_{\mbox{\tiny{MD}}}$) and experimental data~($\kappa_{\mbox{\tiny{E}}}$), at a 
given temperature, $T$. PC representation of $\epsilon_{\mbox{\tiny{d}}}$ is given as:

\be
\epsilon_{\mbox{\tiny{d}}} \approx \mathcal{\epsilon}_{\mbox{\tiny{d}}}^{\mbox{\tiny{PCE}}} = 
\sum_{\alpha\in\mathcal{I}} c_{\alpha}(T)\Psi_{\alpha}(\bm{\xi(\theta)}) 
\ee

\noindent The set of parameters in $\bm{\theta}$ are parameterized in terms of canonical random 
variables, $\bm{\xi}$ distributed uniformly in the interval $[-1,1]$. 
 $\Psi_{\alpha}$'s are multivariate polynomial basis functions, orthonormal with respect to the joint probability 
 distribution of $\bm{\xi}$. The degree of truncation in the above expansion is denoted by $\alpha$, a subset of
 the multi-index set $\mathcal{I}$ that comprises of individual degrees of univariate polynomials in $\Psi_{\alpha}$.
The PC coefficients, $c_{\alpha}$'s can be estimated using either numerical quadrature or advanced techniques
such as those involving basis pursuit de-noising~\cite{Peng:2014}, compressive
sampling~\cite{Hampton:2015}, and least angle regression~\cite{Blatman:2011} suited for large-dimensional
applications. However, in our case, since the response surface is 2D, we use Gauss-Legendre quadrature to
obtain accurate estimates of the PC coefficients. 
\bigskip

In order to construct the response surface of the discrepancy, we consider respective intervals for the Si bar length,
$L$ and the applied thermal gradient, $\frac{dT}{dz}$ as $[50a,100a]$~($\angstrom$) and
 $[\frac{1.5}{a},\frac{2.5}{a}]$~($\frac{\mbox{\tiny{K}}}{\tiny{\angstrom}}$); $a$ being the lattice constant. The 
 discrepancy between $\kappa_{\mbox{\tiny{MD}}}$ and $\kappa_{\mbox{\tiny{E}}}$ is computed at the
 Gauss-Legendre quadrature nodes as illustrated in Figure~\ref{fig:rs1}(a) 
  along with the spectrum of PC coefficients in Figure~\ref{fig:rs1}(b). Note that the value of $\kappa_{\mbox{\tiny{E}}}$ was considered to be 149~W/m/K as provided 
 in~\cite{Shanks:1963}. Response surfaces constructed at the bulk temperature, $T$ = 300~K and 500~K are 
 illustrated
 in~\ref{fig:rs2}(a) and~\ref{fig:rs2}(b) respectively. As expected, the discrepancy is observed to decrease
 with the bar length ($L$) as a result of increase in the mean free path. It is however interesting to note that the 
 variation in discrepancy due to changes in the applied thermal gradient in the considered range is found to be
 negligible. The accuracy of the response surfaces is verified by computing a relative L-2 norm of the error
 ($\varepsilon_{\tiny{\mbox{L-2}}}$) on Sobol samples~\cite{Saltelli:2010} (see Figure~\ref{fig:rs2}(c)) in the 2D 
 parameter domain as follows:
 
 \be
 \varepsilon_{\tiny{\mbox{L-2}}} = \frac{\left[\sum_{j}(\epsilon_{\tiny{\mbox{d},j}}^{\mbox{\tiny{MD}}} - 
 \epsilon_{\tiny{\mbox{d},j}}^{\mbox{\tiny{PCE}}})^2\right]^{\frac{1}{2}}}{\left[\sum_j(\epsilon_{\tiny{\mbox{d},j}}
 ^{\mbox{\tiny{MD}}})^2\right]^{\frac{1}{2}}} 
 \label{eq:err_l2}
 \ee
 
 A response surface was also constructed at $T$ = 1000~K (plot not included for brevity) and the impact of
 varying thermal gradient on the discrepancy was still observed to be negligible. In all cases,
 $\varepsilon_{\tiny{\mbox{L-2}}}$ in Eq.~\ref{eq:err_l2} was estimated to be of $\mathcal{O}(10^{-3})$ thereby
 indicating that the response surfaces could be used to predict the discrepancy for a given
 point ($L$,$\frac{dT}{dz}$) in the considered domain with reasonable accuracy. As an additional verification
 step, we plot the inverse of thermal conductivity against the inverse of bar length using data from NEMD
 simulations as well as predictions from the response surface as estimated using Eq.~\ref{eq:rs_line}, 
 in Figure~\ref{fig:rs3}:
 
 \be
\kappa_{\mbox{\tiny{MD}}} =  \kappa_{\mbox{\tiny{E}}} - \epsilon_{\mbox{\tiny{d}}} 
 \label{eq:rs_line}
 \ee
 
 \noindent where $\epsilon_{\mbox{\tiny{d}}}$ is estimated from the response surface. 
 It is observed that the response surface estimates exhibit an expected linear trend, consistent with NEMD
  predictions. Constructing response surfaces using a relatively small number of NEMD predictions thus offers
  potential for huge computational savings for studies aimed at predicting thermal conductivity trends and 
  quantifying the discrepancy with measurements for a wide range of systems sizes, applied thermal gradients as 
  well as bulk temperatures. 
 




































