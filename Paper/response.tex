\section{Discrepancy with Experiments}
\label{sec:response}

As discussed earlier in Section~\ref{sec:intro}, bulk thermal conductivity estimates using NEMD simulations
are severely under-predicted primarily due to reduction in mean free path associated with phonon transport. 
Additionally, the introduction of thermostats causes a significant amount of variability in the applied thermal gradient,
especially in their vicinity due to a so-called Kapitza effect. We illustrate this phenomenon by plotting temperature 
distribution along the length of the bar in Figure~\ref{fig:kapitza}. In this section, we focus on the impact of
system size, specifically the length of the Si bar as well as the variability in applied thermal gradient on discrepancy
in bulk thermal conductivity between NEMD predictions and experimental data. For this purpose, we consider
a range of values for the bar length and the thermal gradient. In order to determine the discrepancy trends, one
might consider evaluating the thermal conductivity using NEMD simulations for a large set of values of length and
thermal gradient. However, considering the computational expense associated with each set, this approach quickly
becomes computationally prohibitive. Instead, we construct a response surface using a 2D
Polynomial Chaos (PC) representation of the discrepancy requires NEMD predictions for a small number of
combinations of length and thermal gradient values as discussed in the following section. 

\subsection{Polynomial Chaos}